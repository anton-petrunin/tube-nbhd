\documentclass[a4paper,10pt]{article}

\usepackage{paper-en}
\usepackage{hyperref}
%\usepackage[notref,notcite,color]{showkeys}


\def\thetitle{Tubed embeddings}
\def\theauthors{Anton Petrunin}

\hypersetup{colorlinks=true,
citecolor=black,
linkcolor=black,
anchorcolor=black,
filecolor=black,
menucolor=black,
urlcolor=black,
pdftitle={\thetitle},
pdfauthor={\theauthors}
}

%\overfullrule=100mm

\begin{document}
%\pagestyle{empty}\renewcommand\includegraphics[2][{}]{}


\title{\thetitle}
\author{\theauthors}
\date{}
\maketitle

\begin{abstract}
We consider the following question:
When does a Riemannian manifold admit an embedding with uniformly thick tubular neighborhood in another Riemannian manifold of large dimension?
\end{abstract}

\section*{Introduction}

\paragraph{}\label{par:Observation}
Let $M$ and $N$ be smooth Riemannian manifolds; we will always assume that Riemannian manifolds are complete and connected.

A smooth embedding $f\:M\hookrightarrow N$ will be called \emph{tubed} if the image $f(M)\z\subset N$ is a closed subset and it admits a tubular $\eps$-neighborhood for some $\eps>0$;
that is, the closest-point projection from the $\eps$-neighborhood of $f(M)$ to $f(M)$ is uniquely defined.

We say that $M$ has \emph{uniformly polynomial growth} if there is a polynomial $p$ such that 
\[\vol B(x,R)_M\le p(R)\]
for any point $x\in M$ and any radius $R\ge 0$;
here $B(x,R)_M$ denotes open ball in $M$ of radius $R$ centered at $x$.
In this case we say that $p$ is a \emph{growth polynomial} of $M$.

\begin{thm}{Observation}
Suppose a complete Riemannian manifold $M$ admits an isometric tubed embedding in a Euclidean space.
Then $M$ has
\begin{enumerate}[(i)]
\item bounded sectional curvature,
\item positive injectivity radius, and
\item\label{uniformly polynomial growth} uniformly  polynomial growth.
\end{enumerate}
\end{thm}

For example,
the Lobachevsky plane has exponential growth;
therefore, it does not admit an isometric tubed embedding into $\RR^d$ for any integer $d$.




\paragraph{}\label{par:main} 
Further, the sectional curvature of $M$ will be denoted by $\sec_M$;
its injectivity radius at point $p\in M$ will be denoted by $\inj_pM$, and $\inj M$ will denote the injectivity radius of $M$; that is, $\inj M\df \inf_{p\in M}\{\,\inj_pM\,\}$.

We say that $M$ has \emph{bounded geometry} if $\inj M>0$ and for every $k\in\NN$ there is a constant $C_k > 0$ such that $|\nabla^k \Rm| \le C_k$;
here $\nabla$ denotes the Levi-Civita connection, and $\Rm$ is the curvature tensor on $M$.
Note that manifolds with bounded geometry have bounded sectional curvature.

\begin{thm}{Main theorem}
Suppose $M$ is a Riemannian manifold with bounded geometry.
Then, $M$ has uniformly  polynomial growth
if and only if there is an isometric tubed embedding $M \hookrightarrow \RR^d$ for some integer $d$.

Moreover the dimension $d$ can be found in terms of the dimension and a growth polynomial of $M$.
\end{thm}

\paragraph{Other ambient spaces.}\label{par:other-intro}
Loosely speaking, the following statement says that there is no universal space --- no space can serve as a target of an isometric tubed embedding for \textit{any} $n$-manifolds with bounded geometry.

\begin{thm}{Proposition}
Let $N$ be a Riemannian manifold with bounded geometry and $n\ge 2$.
Then there is an $n$-dimensional  Riemannian manifold $M$ with bounded geometry that does not admit an isometric tubed embedding in $N$.
\end{thm}

\paragraph{About the proofs.}
The observation in §\ref{par:Observation} is proved in §§\ref{lem:gauss}--\ref{par:obs-proof}.
It is done by means of elementary differential geometry.

To prove the main theorem (§\ref{par:main}), we approximate the manifold by a graph and apply a slight modification of 
the theorem by Robert Krauthgamer, and James Lee \cite{krauthgamer-lee0,krauthgamer-lee1}.
It produces a map $f_1\:M\to \RR^{d_1}$ such that $|f_1(x)-f_1(y)|$ is bounded away from zero for all pairs $x,y\in M$ on distance at least $R$ from each other for some fixed $R>0$.

Further, we construct anther map $f_2\:M\to \RR^{d_2}$ that is bi-Lipschitz in every ball of radius $R$.
The direct sum $f_1\oplus f_2$ is an embedding.
We can choose $\eps>0$ such that $\eps^2\cdot h<\tfrac12\cdot g$,
where $h$ is the induced Riemannian metric on $M$ by $f_1\oplus f_2$ and $g$ is the original Riemannian metric on $M$.
Applying Nash's embedding theorem for metric $g-\eps^2\cdot h$,
we find another map $f_3\:M\to \RR^{d_3}$ such that the direct sum $f=(\eps\cdot f_1)\oplus (\eps\cdot  f_2)\oplus f_3$ is isometric.

We show that the constructions of $f_1$, $f_2$, and $f_3$ produce \textit{uniformly smooth} maps; see  §\ref{par:rough-embedding}.
In particular, the normal curvatures of $f$ are bounded.
Then we use elementary differential geometry to show that $f$ is tubed. 

The proof of the proposition in §\ref{par:other-intro} is a slight modification of the argument of Florian Lehner \cite[1.2]{lehner} (see also \cite{462670}); it states that \textit{no connected graph with bounded degree contains a copy of any connected graph of degree 3}.

\section*{Observation}

\paragraph{Lemma.}\label{lem:gauss}
\textit{Suppose $M$ is a smooth submanifold in $\RR^d$ with normal curvatures at most 1.
Then $-2\le \sec_M(\sigma)\le 1$ for any sectional direction $\sigma$ of $M$.}
\medskip

\parit{Proof.}
Denote by $\T_p$ and $\N_p$ the tangent and normal spaces at $p\in M$.
Denote by $s$ the second fundamental form of $M$ at $p$;
it is a symmetric quadratic form on $\T_p$ with values in $\N_p$.

We need to show that if $|s(v,v)|\le 1$ for any $|v|=1$, then curvature of $M$ at $p$ in any sectional direction
lies in the range $[-2,1]$.
Passing to a 2-dimensional subspace of $\T_p$, we may assume that $\dim M=2$.

Let $h\in \N_p$ and $\zh\in\RR$ be the average values of $s(u,u)$ and $|s(u,u)|^2$ for unit vectors $u\in \T_p$.
Since $\dim M=2$, the Gauss formula \cite[2.1]{petrunin2023} can be written as
\[K=3\cdot |h|^2-2\cdot \zh;\]
here $K$ is the Gauss curvature of $M$ at $p$.
Note that $|h|^2\le \zh$.
Since the normal curvatures are at most 1, we have $\zh\le 1$.
Hence the lemma follows.
\qeds

It is straightforward to check that the inequality $0\le |h|^2 \le \zh$ is the only restriction on $|h|^2$ and $\zh$.
Therefore, the obtained bounds are optimal.

\paragraph{Proof of the observation in §\ref{par:Observation}.}\label{par:obs-proof}
Suppose  a smooth $n$-dimensional submanifold $M$ in $\RR^d$ has a $\eps$-thick tubular neighborhood $W$.

Note that normal curvatures of $M$ cannot exceed $\tfrac1\eps$.
By §\ref{lem:gauss}, 
\[-\tfrac2{\eps^2}\le\sec_M\le\tfrac1{\eps^2}.\]

\begin{wrapfigure}{o}{25 mm}
\vskip-2mm
\centering
\includegraphics{mppics/pic-5}
\end{wrapfigure}

Let us show that $\inj M\ge\eps$; suppose $\inj_pM<\eps$ for some~$p$.
Since $\sec_M\le\tfrac1{\eps^2}$,
Klingenberg's lemma \cite[5.6]{cheeger-ebin} implies that there is a geodesic loop $\gamma$ based at $p$ with length $2\cdot \inj_pM$.
Let \[r=\max\set{|p-q|_{\RR^d}}{q\in\gamma}.\]
Note that $r<\inj_pM$, and $\gamma$ has curvature at least $\tfrac1r$ at $q$.
It follows that the normal curvature of $M$ at $q$ in the direction of $\gamma$ exceeds $\tfrac1\eps$ --- a contradiction.

Consider the closest point projection $\pi\:W\z\to M$.
Note that there is a constant $c\z=c(\eps,n,d)>0$ such that
\[\vol_d (\pi^{-1}X)\ge c\cdot\vol_n X\]
for any measurable set $X\subset M$.

{

\begin{wrapfigure}{o}{45 mm}
\vskip-5mm
\centering
\includegraphics{mppics/pic-10}
\vskip5mm
\end{wrapfigure}

Note that $B(x,R)_M\subset B(x,R)_{\RR^d}$ for any $x\in M$;
here $B(x,R)_M$ and $B(x,R)_{\RR^d}$ denote balls in the metric of $M$ and $\RR^d$ respectively.
Therefore 
\[\pi^{-1}[B(x,R)_M]\subset B(x,R+\eps)_{\RR^d}.\]
Since $R\mapsto \vol_d B(x,R+\eps)_{\RR^d}$ is a polynomial function, we get that $M$ has uniformly  polynomial growth.
\qeds

}

\section*{Graph embedding}

\paragraph{}\label{par:graph-embedding}
Let $\Gamma$ be a connected graph.
Recall that $\Gamma$ comes with the shortest-path distance on its vertex set $\Vert \Gamma$.
So, we can talk about ball $B(x,R)_\Gamma$ for any center $x\in \Vert \Gamma$ and radius $R>0$.
The number of vertices in $B(x,R)_\Gamma$ will be denoted by $|B(x,R)_\Gamma|$.
We say that $\Gamma$ has \emph{uniformly polynomial growth} if there is a polynomial $p$ such that 
\[|B(x,R)_\Gamma|\le p(R)\]
for any point $x\in \Gamma$ and any radius $R\ge 0$.
In this case, we say that $p$ is a \emph{growth polynomial} of $\Gamma$.
Note that any graph with uniformly polynomial growth has bounded degree.

We will write $v\sim w$ if two vertices $v,w\in \Vert \Gamma$ are adjacent.
Let us denote by $\ZZ^d_\infty$ the graph with $\Vert \ZZ^d_\infty=\ZZ^d\subset \RR^d$ and
\[v\sim w \iff  \|v-w\|_\infty=1.\]
The next statement follows from \cite[Theorem 5.5]{krauthgamer-lee1}.

\begin{thm}{Graph-embedding theorem}\label{thm:graph-embedding}
Suppose $\Gamma$ is a connected graph with uniformly polynomial growth.
Then, $\Gamma$ is isomorphic to a subgraph of $\ZZ^d_\infty$ for some integer~$d$.
Moreover, the dimension $d$ can be found in terms of growth polynomial of~$\Gamma$.
\end{thm}

\paragraph{}\label{par:Phi}
Let us denote by $\Conv X$ the convex hull of the subset $X\subset \RR^d$.

\begin{thm}{Corollary}\label{cor:graph-embedding}
Suppose $\Gamma$ is a connected graph with uniformly polynomial growth.
Then for some $\rho>0$ and integer $d>0$ there is a map $\Phi\:\Vert \Gamma\to \RR^d$ such that
\begin{enumerate}[(a)]
\item if $v\sim w$, then $|\Phi(v)-\Phi(w)|\le \rho$, and 
\item if $V$ and $W$ are disjoint cliques in $\Gamma$, then the minimal distance from $\Conv[\Phi(V)]$ to $\Conv[\Phi(W)]$ is at least $1$.
\end{enumerate}
Moreover, the dimension $d$ and the distance $\rho$ can be found in terms of a growth polynomial of~$\Gamma$.
\end{thm}

\parit{Proof.}
By §\ref{par:graph-embedding}, it is sufficient to prove the corollary for $\Gamma=\ZZ^n_\infty$.

Take $d=n+2^n$.
Let us color vertices of $\ZZ^n_\infty$ in $2^n$ colors so that adjacent vertices get different colors.
Let $e_1,\dots s_{2^n}$ be the standard basis in $\RR^{2^n}$.
Consider the map $\Phi\:\ZZ^n\to \RR^d=\RR^n\oplus \RR^{2^n}$ defined by $\Phi\:v\mapsto v\oplus e_{i(v)}$ where $i(v)$ denotes the color of $v$.
After an appropriate rescaling, the obtained map meets the conditions.
\qeds

\section*{Approximation by graph}

\paragraph{Intersection graph.}\label{par:intersection-graph}
Let $M$ be a complete connected Riemannian manifold.
Choose $r>0$ and a maximal (with respect to inclusion) set $V\subset M$ of points on distance at least $r$ from each other.
Given $v\in V$, set $B_v=B(v,r)_M$ and $\lambda\cdot B_v=B(v,\lambda\cdot r)_M$.
Note that
\textit{
\begin{enumerate}[(i)]
\item\label{item:1/2B} the balls $\{\tfrac12\cdot B_v\}_{v\in V}$ are disjoint;
\item $\{B_v\}_{v\in V}$ is an open cover of $M$;
\item $\{2\cdot B_v\}_{v\in V}$ is an open cover of $M$ with Lebesgue number~$r$.
\item If $M$ has bounded geometry, then for any positive real $\lambda$, there is an integer $N_\lambda$ such that any $\lambda\cdot r$-ball in $M$ contains at most $N_\lambda$ points of $V$.
\end{enumerate}
}

Further assume that $M$ has bounded geometry.
Let $\Gamma_\lambda$ be the intersection graph of the covering $\{\lambda\cdot B_v\}$ for $\lambda\ge 1$;
that is, $\Vert \Gamma_\lambda=V$, and 
\[v\sim w\quad\iff\quad \lambda\cdot B_v\cap \lambda\cdot B_w \ne \emptyset;\]
here $v\sim w$ means that $v$ is adjacent to $w$ in $\Gamma_\lambda$.
Observe that $\Gamma_\lambda$ has degree at most~$N_{2\cdot\lambda}-1$, and 
\[\lambda\cdot r\cdot |v-w|_{\Gamma_\lambda}\ge |v-w|_M\eqlbl{eq:dist-G}\]
for any $v,w\in V$.

Since $M$ has bounded geometry, there is $\delta>0$ such that $\vol (\tfrac12\cdot B_v) >\delta$ for any $v\in V$.
Therefore, \textit{\ref{item:1/2B}} and \ref{eq:dist-G} implies the following:

\textit{
\begin{enumerate}[(i)]\addtocounter{enumi}{4}
\item If $M$ has bounded geometry and uniformly polynomial growth, then any $\lambda\ge1$, the graph $\Gamma_\lambda$ has uniformly polynomial growth.\\
Moreover, a growth polynomial of $\Gamma_\lambda$ can be found in terms of $\lambda$, $r$ and a a growth polynomial of $M$.
Namely, if $x\mapsto p(x)$ is a growth polynomial of $M$, then $x\mapsto p(\lambda\cdot r\cdot x)$ is a growth polynomial of $\Gamma_\lambda$.
\end{enumerate}
} 

\section*{Factors of embedding}

\paragraph{}\label{par:rough-embedding}
Let $M$ be a smooth Riemannian manifold.
A map $f\:M\to\RR^d$ is called \emph{uniformly smooth} if 
for any integer $k\ge 1$ there is a constant $C_k$ such that  
\[|\tfrac{d^k}{dt^k}(f\circ\gamma)(t)|\le C_k\] 
for all unit-speed geodesics $\gamma$.

\begin{thm}{Proposition}
Let $M$ be a complete $n$-dimensional Riemannian manifold with bounded geometry and uniformly  polynomial growth.
Suppose $|\sec_M|\le \tfrac1{100}$ and $\inj M\ge 10$.
Then for some integer $d_1$ there is a uniformly smooth map $f_1\:M\z\to\RR^{d_1}$ such that 
\[|x-y|_M\ge 1
\quad\Longrightarrow\quad
|f_1(x)-f_1(y)|_{\RR^{d_1}}\ge 1.\]
Moreover, the dimension $d_1$ can be found in terms of a growth polynomial of~$M$.
\end{thm}

\parit{Proof.}
Let us apply the construction in §\ref{par:intersection-graph} for $r=\tfrac14$;
we obtain a subset $V\subset M$, balls $\{\lambda\cdot B_v\}_{v\in V}$, integer values $N_\lambda$ and connected graphs $\Gamma_\lambda$ with $\Vert\Gamma_\lambda =V$ for all $\lambda\ge 1$.

Now, let us apply the construction of partition of unity.
Choose a $C^\infty$-smooth function $\sigma\:\RR\to [0,1]$
such that $\sigma(t)=1$ for $t\le 1$ and $\sigma(t)=0$ for $t\ge 2$.
Further, let $\psi_v=\sigma(\tfrac{\dist_v}{r})$, for each $v\in V$;
note that the functions $\phi_v$ are uniformly smooth.
Set 
\[\Psi=\sum_{v\in V} \psi_v\quad \text{and}\quad \phi_v=\frac{\psi_v}{\Psi}.\]
The obtained functions $\phi_v$ form a partition of unity subordinate to $\{2\cdot B_v\}$.
Note that $1\z\le \Psi\z\le N_2$; therefore, the functions $\phi_v$ are uniformly smooth. 

Given $x\in M$, consider the set 
\[V_x=\set{v\in V}{\phi_v(x)>0}.\]
Note that $\phi_v(x)>0$ implies that $2\cdot B_v\ni x$.
Therefore $V_x$ is a clique in $\Gamma_2$.
Moreover,
\[|x-y|\ge 4\cdot r=1
\qquad\Longrightarrow\qquad
V_x\cap V_y=\emptyset.
\eqlbl{eq:cliques}
\]


Let $\Phi\: V\to\RR^{n}$ be the map constructed in §\ref{par:Phi}.
Given $x\in M$, define $f_1(x)$ as the barycenter of points $\Psi(v)$ with masses $\phi_v(x)$ for all $v\in V_x$.
By §\ref{par:Phi}, $\diam V_x\le \rho$,
and since the functions $\phi_v$ are uniformly smooth, so is $f_1$.
Further, note that $f_1(x)$ lies in the convex hull of $\Phi(V_x)$ for any $x\in M$.
Finally \ref{eq:cliques} and §\ref{par:Phi} imply the proposition.
\qeds

\paragraph{Proposition.}\label{par:local-embedding}
\textit{
Let $M$ be a complete $n$-dimensional Riemannian manifold with bounded geometry.
Suppose $|\sec_M|\le \tfrac1{100}$ and $\inj M\ge 10$.
Then for some integer $d_2$, there is a uniformly smooth map $f_2\:M\to\RR^{d_2}$ such that 
$f_2$ is uniformly bi-Lipschitz in all unit balls.
}

\parit{Proof.}
Choose $r=1$ and apply the constructions in §\ref{par:intersection-graph}.

Let $\sigma\:\RR\to[0,1]$ be as in §\ref{par:rough-embedding}.
For each $v\in V$, choose an isometry $\iota_v\:\T_v\to\RR^n$.
Consider the map $s_v\:M\to \RR^n$ defined by
\[s_v\:\exp_vx\mapsto \sigma(2\cdot |x|)\cdot \iota_v(x).\]

Since $\Gamma_2$ has degree at most $N_2-1$,
the set $V$ can be subdivided in $N_2$ subsets $V_1,\dots, V_{N_2}$ such that 
for any $i$, the balls $2\cdot B_v$ for $v\in V_i$ are disjoint.
Consider the map $S_i\:M\to\RR^n$ defined by
\[S_i\df\sum_{v\in V_i}s_v.\]
Note that $S_i$ is uniformly smooth and it is uniformly bi-Lipschitz in every ball $2\cdot B_v$ for $v\in V_i$.
It follows that $S_1\oplus\dots\oplus S_{N_2}\:M\to \RR^{N_2\cdot n}$ is uniformly smooth and bi-Lipschitz in  $2\cdot B_v$ for any $v\in V$.
The proposition follows since $r$ is a Lebesgue number of the covering $\{2\cdot B_v\}$.
\qeds

\section*{Uniform Nash's construction}

\paragraph{Uniformly regular metrics.}
\label{par:uniformly-regular}
Let $M$ be a smooth $n$-dimensional manifold with a chosen locally finite atlas $\mathcal{A}$.
We say that $\mathcal{A}$ is \emph{appropriate} if it meets the following conditions:
\begin{itemize}
\item Each chart $\sigma\in \mathcal{A}$ is given by a map from an open set $U_\sigma\subset M$ to the unit ball $\DD^n\subset \RR^n$.
\item The covering $\{U_\sigma\}$ of $M$ has bounded multiplicity.
\item There exists $r<0$ such that $\{\sigma^{-1}(r\cdot \DD^n)\}_{\sigma\in\mathcal{A}}$ is a cover for $M$.
\item All the transition maps of $\mathcal{A}$ are uniformly bi-Lipschitz and for any integer $k\ge 0$ they  have uniformly bounded $C^k$-norms. 
\end{itemize}

A metric tensor $g$ on $M$ will be called \emph{uniformly regular} with respect to an appropriate atlas $\mathcal{A}$
if for every integer $k$ all components of the metric tensor in all charts of $\mathcal{A}$ have bounded $C^k$-norms,
and all the charts are uniformly bi-Lipschitz with respect to the metric induced by $g$ on $M$. 

Analogously, we can define a \emph{uniformly regular sequence of metrics} on a fixed manifold with a chosen appropriate atlas.
(Note that a smooth metric on a fixed compact manifold, say $\mathbb{S}^n$, is uniformly smooth for some appropriate atlas,
but a sequence of smooth metrics does not have to be uniformly smooth.)

Let $M$ be a Riemannian manifold with bounded geometry.
Applying rescaling if necessary, we may assume $|\sec_M|\le \tfrac1{100}$ and $\inj M\ge 10$.

Let us apply the construction in §\ref{par:intersection-graph} for $r=\tfrac12$.
Given $v\in V$, consider the normal chart $\sigma_v\:2\cdot B_v\to\DD^n$ with the center at $v$;
we will call such an atlas \emph{standard}.
It is straightforward to check that this atlas is appropriate and
the metric tensor is uniformly regular with respect to this atlas.
In particular, \textit{any manifold of bounded geometry has a uniformly regular metric in an appropriate atlas}.
The following statement from \cite{disconzi-shao-simonett} implies that the converse holds as well.

\begin{thm}{Proposition}
A Riemannian manifold $M$ has bounded geometry if and only if its metric is uniformly regular for some appropriate atlas.
\end{thm}

Note that if the atlas on $M$ meets the conditions in the proposition,
then a map $f\:M\to \RR^d$ is uniformly smooth if and only if for every integer $k\ge 0$,
the functions $f\circ\sigma-f\circ\sigma(0)\: \DD^n\to \RR^d$ for all charts $\sigma\in \mathcal{A}$
have uniformly bounded $C^k$-norms.


\paragraph{Proposition.}\label{par:nash}
\textit{Let $M$ be a complete $n$-dimensional Riemannian manifold with bounded geometry.
Then for some integer $d$ there is a uniformly smooth isometric immersion $f_3\:M\to\RR^d$.}

\medskip

The statement follows from the proof of Nash's theorem about regular embedding;
we will describe the needed changes in \cite{nash}.

Suppose we have a sequence of Riemannian metrics $g_1,g_2,\dots $ on the sphere $\mathbb{S}^n$.
By Nash's theorem, there are smooth embeddings $s_1,s_2,\z\dots\:\mathbb{S}^n\to \RR^d$
such that each $s_i$ induces $g_i$ on $\mathbb{S}^n$; here $d\z=\tfrac12\cdot n\cdot(3\cdot n+11)$.

Suppose that the sequence of metrics $g_i$ is uniformly regular with respect to a finite appropriate atlas on $\mathbb{S}^n$.
Then the construction in \cite[Part C]{nash} with the same choices of functions produces a uniformly smooth embeddings $s_i$.
Namely, the functions $\psi^r$ in \cite[(C2)]{nash} and $\alpha_r$ in \cite[(C10)]{nash} are uniformly smooth and we can assume that $\lambda$ in \cite[(C12)]{nash} is the same for all $g_i$. 

Further, the so-called \emph{Nash's reduction} \cite[Part D]{nash} reduces the existence of an isometric immersion of $M$ to the existence of isometric immersions $(\mathbb{S}^n,g_i)\to \RR^d$ for a sequence of Riemannian metrics $g_i$.
Nash starts with a locally finite atlas on a manifold and produces a sequence of metrics $g_i$ on $\mathbb{S}^n$ with a sequence of smooth maps $\phi_i\:M\to (\mathbb{S}^n,g_i)$ 
such that the sum of metrics induced by $\phi_i$ on $M$ coincides with $g$.

Applying rescaling, we may assume that $|\sec_M|\le \tfrac1{100}$ and $\inj M\ge 10$.
Suppose that $\{\lambda\cdot B_v\}_{v\in V}$ as in §\ref{par:rough-embedding}.
If the atlas is chosen from normal coordinates for balls $\{2\cdot B_v\}_{v\in V}$, then Nash's construction produces a uniformly smooth sequence of maps $\phi_i$ and a sequence of metrics satisfying the above condition.
As a result, we get a uniformly smooth isometric immerstion $(M,g)\to \RR^d$, where $d= \tfrac12\cdot N_2\cdot n\cdot(3\cdot n+11)$.

\section*{Proof assembling}


\paragraph{Proof of the main theorem (§\ref{par:main}).}\label{par:main-proof}
Let $f_1\:M\to\RR^{d_1}$ and $f_2\:M\to\RR^{d_2}$ be the maps constructed in §\ref{par:rough-embedding} and §\ref{par:local-embedding}.
Consider the map $f_1\oplus f_2\:M\to \RR^{d_1+d_2}$ defined by $p\mapsto (f_1(p),f_2(p))\in \RR^{d_1}\oplus\RR^{d_2}=\RR^{d_1+d_2}$.

Let $h$ be the metric induced by $f_1\oplus f_2$
and $g$ be the original metric on $M$.

Choose $\eps>0$ such that $\eps^2\cdot h<\tfrac12\cdot g$.
Note that $g-\eps^2\cdot h$ is a uniformly regular and smooth metric on $M$ for a standard atlas.
By §\ref{par:uniformly-regular}, $(M,g-\eps^2\cdot h)$ has bounded geometry.
By §\ref{par:nash} there is a uniformly smooth map $f_3\:M\to\RR^{d_3}$ that induces metric $g-\eps^2\cdot h$ on $M$.

Therefore, the map $f=(\eps\cdot f_1)\oplus (\eps\cdot f_2)\oplus f_3\:M\to \RR^{d_1+d_2+d_3}$ is a uniformly smooth isometric immersion.
In particular, the image $f(M)$ has bounded normal curvatures.
By §\ref{par:rough-embedding} and §\ref{par:local-embedding}, $f_1\oplus f_2$ is an embedding;
therefore so is $f$.
It remains to show that $f$ is tubed.


\begin{wrapfigure}{o}{45 mm}
\vskip-0mm
\centering
\includegraphics{mppics/pic-20}
\end{wrapfigure}

Assume the contrary;
that is, for any $\eps>0$ there is a point $x\in \RR^d$ that lies on distance smaller than $\eps$ to $f(M)$ such that the distance function from $x$ has at least two minimum points on $f(M)$,
say $f(p)$ and $f(q)$.
Denote by $\gamma$ a minimizing geodesic from $p$ to $q$ in $M$.

Choose $\eps<\tfrac12$.
The properties in §\ref{par:rough-embedding} and §\ref{par:local-embedding} imply that 
\[\length\gamma<\frac\eps\delta\]
for some fixed $\delta>0$.
Let $y\in f(\gamma)$ be a point that maximizes the distance from $x$, and $r=|x-y|$.
The triangle inequality implies that $r\le \eps\cdot(1+\tfrac1\delta)$.
It follows that the curvature of $\gamma$ at $y$ is at least $\tfrac1r\ge \tfrac1\eps\cdot \tfrac\delta{1+\delta}$.
The latter is impossible if $\eps$ is small.
\qeds

\section*{Universal space}

\paragraph{Theorem.}\label{par:univeral}\textit{
Let $\Gamma$ be a connected graph of bounded degree.
Then there is a connected graph $\Delta$ of degree $\le 3$ such that there is no map $f\:\Vert \Delta\to \Vert \Gamma$ such that
\begin{enumerate}[(i)]
\item $|f^{-1}(h)|\le K$ for any vertex $h$ in $\Gamma$, and
\item $|f(v)-f(x)|_\Gamma\le K$ for any two adjacent vertices $v, w$ in $\Delta$
\end{enumerate}
for some constant $K$.
}

\medskip

A slightly weaker statement was proved by Florian Lehner \cite[1.2]{lehner} and rediscovered by Fedor Petrov \cite{462670};
we will modify their proof straightforwardly.

\parit{Proof.}
Suppose $\Gamma$ is an infinite connected graph with bounded degree.
Note that $\Gamma$ has a countable set of vertices; so we can identify $\Vert \Gamma$ with the set of natural numbers $\NN$.

Choose a sequence $n_1,n_2,\dots$ of natural numbers such that each $n\in \NN$ appears infinitely many times in it;
for example, we may take the following sequence
\[1,1,2,1,2,3,1,2,3,4,1,\dots\]

Let us construct a graph $\Delta$ with degree at most $3$ such that no map $f$ meets the condition in the theorem for any $K$.
In the following construction, the set of vertices of $\Delta$ will be $\NN$, and it will contain an infinite path $P=1\sim 2\sim \dots$.
There will be a fast growing sequence $S_1<S_2<\dots$ such that 
if $i\sim j$ for some $i<j$, then $j-i=1$ or $S_k<i<j\le S_{k+1}$ for some $k$.

The subgraphs induced by $\{1,\dots, S_k\}$ will be denoted by $\Delta_k$;
let $P_k$ be the corresponding path $1\sim 2\sim \dots \sim S_k$.
The graphs $\Delta_k$ will be constructed recursively so that they meet the following condition:
there is no map $f_k\:\Vert \Delta_k\to \Vert \Gamma$ that meets the condition for $K=k$ and such that $f_k(1)\z=n_k$.
Once it is done, the theorem follows. 

We assume that $\Delta_0$ is the empty graph, so $S_0=0$.
Assume $\Delta_{k-1}$ is constructed, let us  construct $\Delta_k$.
To do so, we need to choose large $S_k\z\gg S_{k-1}$ and add diagonals between the vertices in the integer interval
\[\{S_{k-1}\z+1,\dots,S_k\}.\]

Suppose $L<(S_k-S_{k-1})/2$.
Let $A$ ($B$) be the first (respectively, the last) $L$ elements in the integer interval.
We may choose an arbitrary bijection $A\leftrightarrow B$ and connect corresponding vertices by edges.
This way we get $L!$ different graphs; in particular, the total number of different graphs $\Delta_k$ that can be obtained this way has superexponential growth with respect to $S_k$.
More precisely, we count labeled graphs; that is, graphs $\Delta_k$ with vertices labeled by $\{1,\dots,S_k\}$ and a path $P_k=1\sim 2\sim\dots\sim S_k$.

How much of these graphs have the required maps $f\:\Vert \Delta_k\to \Vert \Gamma$?
Suppose that the degree of $\Gamma$ does not exceed $d-1$.
Note that each $k$-ball in $\Gamma$ has at most $d^k$ vertices.
Therefore we have at most $d^{k\cdot S_{k+1}}$ different maps of the path $P_k$.
Given one such map, there are at most $k\cdot d^k$ ways to add a diagonal at each vertex.
Therefore the number of labeled graphs that admit a required map $f\:\Vert \Delta_k\to \Vert \Gamma$ for $K=k$ grows exponentially with respect to $N_k$.
It follows that we may choose a sufficiently large $N_k$,
so that there is a graph $\Delta_k$ that does not admit the map $f_k$ described above.
\qeds

\paragraph{Proof of the proposition in §\ref{par:other-intro}.}\label{par:other-proof}
Let $N$ be a manifold with bounded geometry and $n\ge 2$;
we need to construct an $n$-dimensional manifold $M$ with bounded geometry that does not admit an isometric tubed embedding into $N$.

Let $\Gamma_2$ be the intersection graph obtained by applying the construction in §\ref{par:intersection-graph} to $N$.
Recall that $\Gamma_2$ has bounded degree.
Let $\Delta$ be the graph provided by applying the theorem in §\ref{par:univeral} to $\Gamma_2$;
it is a connected graph with  degree at most~3.

Let us choose a \emph{tube}; that is, the cylinder $\mathbb{S}^{n-1}\times [0,1]$ with smooth Riemannian metric of such that a neighborhood of its boundary components are isometric to an small annulus in the unit sphere $\mathbb{S}^n$, so it can be used to connect spheres.
Prepare a unit sphere $\mathbb{S}^n$ for every vertex in $\Delta$.
Connect two spheres by a tube if the corresponding vertices are adjacent.
Note that the obtained manifold $M$ has bounded geometry.

Assume there is an isometric tubed embedding $\iota\colon M\hookrightarrow N$.
We can assume that $\Delta$ is embedded in $M$ so that each vertex lies on the corresponding sphere.
Consider the map $\Delta\to \Gamma_2$ that sends a vertex $v\in \Vert \Delta$ to a vertex of $\Gamma_2$ that lies on minimal distance from $\iota(v)$.
Note that for some $K$, the obtained map satisfies the conditions in the theorem in §\ref{par:univeral} --- a contradiction.
\qeds





\section*{Final remarks}


\paragraph{}\label{par:remarks}

In the definition of uniformly regular metric,
instead of bounded $C^k$ norm for all $k$,
one may require bounded $C^{k,\alpha}$ norm for fixed $k$ and $\alpha$.
Let us call such metrics \emph{uniformly $C^{k,\alpha}$-regular}.

If we follow the argument in §\ref{par:nash}, using Günther's version of the embedding theorem \cite{guenther},
then we obtain the following: \textit{for any $k\ge 2$ and $\alpha\in(0,1)$, any manifold with $C^{k,\alpha}$-regular metric admits a uniformly $C^{k,\alpha}$-smooth isometric immersion into Euclidean space;}
in particular, such an immersion has bounded normal curvatures.
Further, applying this result and following the argument in §\ref{par:other-proof}, we get that
\textit{for any $k\ge 2$ and $\alpha\in(0,1)$, any manifold with $C^{k,\alpha}$-regular metric and uniformly polynomial growth admits a tubed isometric embedding into a Euclidean space of large dimension.}
The following question remains open.

\begin{thm}{Question}
Is it true that any manifold with uniformly $C^{1,1}$-regular metric and uniformly polynomial growth admits a tubed isometric embedding into a Euclidean space of large dimension?
\end{thm}

This question is interesting since a manifold with $C^{1,1}$-regular metric has bounded sectional curvature and positive injectivity radius.
It is unknown if the converse holds.
But if it does, and the question has an affirmative answer,
then we would get a converse to the observation in 
§\ref{par:Observation}.

Note that an embedding $f\:M\to \RR^d$ with bounded normal curvature is tubed if and only if it is \emph{uniformly graphical};
that is, there is $r>0$ such that for any $p\in M$ the set $B(f(p),r)\cap f(M)$ is a graph of a function $\RR^n\to \RR^{d-n}$ for a choice of coordinate system in $\RR^d$.
If one argues as above, using the result of Anders K\"{a}ll\'{e}n \cite{kallen}, then we get that 
\textit{for any $\alpha\in(0,1)$, any manifold with $C^{1,\alpha}$-regular metric and uniformly polynomial growth admits a uniformly graphical embedding into a Euclidean space of large dimension.}
In other words, the $C^{2,\eps}$ and $C^{1,1-\eps}$ versions of the last question have affirmative answers, but $C^{1,1}$ remains open.

\begin{thm}{Question}
Suppose $M$ is an $n$-dimensional Riemannian manifold of bounded geometry.
What is necessary and sufficient condition for the existence of tubed isometric embeddings of $M$
into hyperbolic space of sufficiently large dimension (or product of sufficiently many hyperbolic planes)?
\end{thm}

Note that a horosphere defines a tubed isometric embedding $\RR^{d-1}\hookrightarrow\HH^d$.
It follows that uniformly polynomial growth is a sufficient condition, which is evidently not necessary.
On the other hand the argument by David Hume and Alessandro Sisto \cite[1.1]{hume-sisto} can be used to find some necessary conditions; for example, \textit{the product of hyperbolic plane and real line does not admit a tubed embedding into hyperbolic space of arbitrary dimension}.

\paragraph{Acknowledgments.} 
This note is inspired by the question of Peter Michor \cite{124840}.
I want to thank 
\texttt{aorq} (an anonymous user of MathOverflow),
Michael Gromov,
Nina Lebedeva,
James Lee,
Alexander Lytchak, and
Danielel Semola,
for help.


{\sloppy
\def\emph{\textit}
\printbibliography[heading=bibintoc]
\fussy
}
\end{document}


It seems that there is currently no technique that provides an approach to the next question.

\begin{thm}{Question}
Suppose $M$ is a complete smooth Riemannian manifold with bounded curvature, positive injectivity radius and polynomial volume growth.
Does it admit an isometric tubed embedding in a Euclidean space of sufficiently large dimension?
\end{thm}

We provided an affirmative answer assuming bounded geometry;
this condition is equivalent to the so-called \emph{uniformly regular} Riemannian metrics \cite{disconzi-shao-simonett};
the latter means that one can choose a locally finite bi-Lipschitz atlas of the manifold such that the domains of the charts have form a covering with positive Lebesgue number  and such that the components of metric tensor in each chart have $C^k$ bounded components for each $k$.
For our proof, it is sufficient to assume that this condition holds for $k\le 3$.
Moreover, if instead of Nash's proof we would use Günther's proof \cite{guenther}, then we could only assume that the metric has bounded $C^{2,\alpha}$ norm for some $\alpha>0$.

The following question seems to be open.

\begin{thm}{Question}
Suppose that a complete smooth Riemannian manifold $M$ has bounded sectional curvature and positive injectivity radius. 
Is it possible to find an atlas on $M$ such that covering by charts has positive Lebesgue number, all charts are uniformly bi-Lipschitz and the components of the metric tens in all charts have uniformly bounded $C^{1,1}$ norm in all charts?
\end{thm}


It seems to be unknown if a manifold with bounded curvature admits a bi-Lipschitz atlas such that the metric tensor has $C^{1,1}$ bounded components.
But even if it is true, it is unknown that manfold with such metric admits a uniformly $C^{1,1}$ smooth isometric embedding; that is, all partial derivetives of $f$ in all the charts are uniformly Lipschitz.

Finally, by the result of Anders K\"{a}ll\'{e}n \cite{kallen}, any compact Riemannian manifold with $C^{1,\alpha}$ metric for $0<\alpha<1$ admits an $C^{1,\alpha}$ smooth isometric embedding.
Applying this result together with Nash's reduction, we get an emebdding a complete manifold with uniformly regular and $C^{1,\alpha}$ smooth metric  

